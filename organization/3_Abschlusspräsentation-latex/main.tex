%url = https://www.overleaf.com/7690515vynnjrmjmtvc
%url = https://git.overleaf.com/7690515vynnjrmjmtvc


\documentclass[compress]{beamer}

\usetheme{Hamburg}

\usepackage[T1]{fontenc}
\usepackage[utf8]{inputenc}

\usepackage{lmodern}
\usepackage{svg}

%\usepackage[english]{babel}
\usepackage[ngerman]{babel}

\usepackage{eurosym}
\usepackage{listings}
\usepackage{microtype}
\usepackage{units}
\usepackage{minted}

\usepackage{lineno}
\usepackage{listings}
\usepackage{color}
\usepackage{caption}
\usepackage{listings}
\usepackage{amsmath}
\usepackage{pxfonts}
\usepackage{tikz}
\usepackage{pgfplots}
\usepackage{pgfplotstable}
\usepackage{filecontents}

\lstset{
	basicstyle=\ttfamily\footnotesize,
	frame=single,
	numbers=left,
	language=C,
	breaklines=true,
	breakatwhitespace=true,
	postbreak=\hbox{$\hookrightarrow$ },
	showstringspaces=false,
	tabsize=4,
	captionpos=b,
	morekeywords={gboolean,gpointer,gconstpointer,gchar,guchar,gint,guint,gshort,gushort,glong,gulong,gint8,guint8,gint16,guint16,gint32,guint32,gint64,guint64,gfloat,gdouble,gsize,gssize,goffset,gintptr,guintptr,int8_t,uint8_t,int16_t,uint16_t,int32_t,uint32_t,int64_t,uint64_t,size_t,ssize_t,off_t,intptr_t,uintptr_t,mode_t}
}
\pgfplotsset{tick label style={font=\tiny\bfseries},
	label style={font=\small},
	legend style={font=\tiny}
}
\pgfplotsset{compat=1.13}
\AtBeginSection[]
{
\begin{frame}
\frametitle{Gliederung (Agenda)}
\tableofcontents[currentsection]
\end{frame}
}

\title{Partikelsimulation}
\author{Oliver Heidmann \& Benjamin Warnke}
\institute{Arbeitsbereich Wissenschaftliches Rechnen\\Fachbereich Informatik\\Fakultät für Mathematik, Informatik und Naturwissenschaften\\Universität Hamburg}
\date{2017-01-12}

\titlegraphic{\includegraphics[width=0.75\textwidth]{logo}}

\begin{document}

\begin{frame}
	\titlepage
\end{frame}

\begin{frame}
	\frametitle{Gliederung (Agenda)}
	\tableofcontents
\end{frame}
    
\section{Autotuneing}
\subsection{Gleichverteilte Eingabe}
\begin{frame}[t]
	\frametitle{Laufzeitmessungen}
	\begin{figure}
		\begin{center}
			\begin{tikzpicture}
			\begin{axis}[
				xmin = 0,
				xmax = 17,
				xtick = {1,2,...,16},
				ymin = 120,
				ymax = 320,
				legend pos=north west,
				ymajorgrids=true,
				grid style=dashed,
				ytick = {120,140,...,320},
				xlabel = $Parameter-Kombination$, 
				ylabel = $Sekunden$]
				\addplot table [domain=1:16, only marks, samples=1000,x=row, y=GRID_average_time, col sep=comma] {times_gleichverteilung.csv};
				\addlegendentry{$GRID$}
				\addplot table [domain=1:16, only marks, samples=1000,x=row, y=GRID_LIST_average_time, col sep=comma] {times_gleichverteilung.csv};
				\addlegendentry{$GRID-LIST$}
			\end{axis}
			\end{tikzpicture}
		\end{center}
	\end{figure}
\end{frame}
\begin{frame}[t]
	\frametitle{Resultierende Entscheidung}
	\begin{itemize}
		\item \textbf{c} $\rightarrow$ cut-off-radius
		\item \textbf{f} $\rightarrow$ cut-off-radius-factor
		\item \textbf{s} $\rightarrow$ start-speed
	\end{itemize}
	\begin{align*}
		2\cdot s < c \cdot (f - 1) - 1
	\end{align*}
	\begin{itemize}
		\item \textbf{true} $\rightarrow$ GRID-LIST
		\item \textbf{false} $\rightarrow$ GRID
	\end{itemize}
\end{frame}
\subsection{Nicht gleichverteilte Eingabe}
\begin{frame}[t]
	TODO
\end{frame}	
\section{Literatur}
\subsection{}
\begin{frame}
	\frametitle{Literatur}
	\begin{itemize}
        \item M-Griebel, S. Knapek, G. Zumbuschm, A. Caglar: Numerische Simulation in der Moleküldynamic. Springer, 2003
		\item D.C Rapaport: The Art of Molecular Dynamics Simulation - 2nd edition, Cambridge University Press, 2004
	\end{itemize}
\end{frame}

\end{document}
