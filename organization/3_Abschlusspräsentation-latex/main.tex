%url = https://www.overleaf.com/7690515vynnjrmjmtvc
%url = https://git.overleaf.com/7690515vynnjrmjmtvc


\documentclass[compress]{beamer}

\usetheme{Hamburg}

\usepackage[T1]{fontenc}
\usepackage[utf8]{inputenc}

\usepackage{lmodern}
\usepackage{svg}

%\usepackage[english]{babel}
\usepackage[ngerman]{babel}

\usepackage{eurosym}
\usepackage{listings}
\usepackage{microtype}
\usepackage{units}
\usepackage{minted}

\usepackage{lineno}
\usepackage{listings}
\usepackage{color}
\usepackage{caption}
\usepackage{listings}
\usepackage{amsmath}
\usepackage{pxfonts}
\usepackage{tikz}
\usepackage{pgfplots}
\usepackage{pgfplotstable}
\usepackage{filecontents}

\lstset{
	basicstyle=\ttfamily\footnotesize,
	frame=single,
	numbers=left,
	language=C,
	breaklines=true,
	breakatwhitespace=true,
	postbreak=\hbox{$\hookrightarrow$ },
	showstringspaces=false,
	tabsize=4,
	captionpos=b,
	morekeywords={gboolean,gpointer,gconstpointer,gchar,guchar,gint,guint,gshort,gushort,glong,gulong,gint8,guint8,gint16,guint16,gint32,guint32,gint64,guint64,gfloat,gdouble,gsize,gssize,goffset,gintptr,guintptr,int8_t,uint8_t,int16_t,uint16_t,int32_t,uint32_t,int64_t,uint64_t,size_t,ssize_t,off_t,intptr_t,uintptr_t,mode_t}
}
\pgfplotsset{tick label style={font=\tiny\bfseries},
	label style={font=\small},
	legend style={font=\tiny}
}
\pgfplotsset{compat=1.13}
\AtBeginSection[]
{
\begin{frame}
\frametitle{Gliederung (Agenda)}
\tableofcontents[currentsection]
\end{frame}
}

\title{Partikelsimulation}
\author{Oliver Heidmann \& Benjamin Warnke}
\institute{Arbeitsbereich Wissenschaftliches Rechnen\\Fachbereich Informatik\\Fakultät für Mathematik, Informatik und Naturwissenschaften\\Universität Hamburg}
\date{2017-01-12}

\titlegraphic{\includegraphics[width=0.75\textwidth]{logo}}

\begin{document}

\begin{frame}
	\titlepage
\end{frame}

\begin{frame}
	\frametitle{Gliederung (Agenda)}
	\tableofcontents
\end{frame}
\section{Parallelisierung(Linked-Cells)}
\subsection{Open MP}
\begin{frame}[t]
	\frametitle{Laufzeitmessungen}
	\begin{figure}
		\begin{center}
			\begin{tikzpicture}
			\begin{axis}[
			xmin = 0,
			xmax = 26,
			xtick = {2,4,6,...,24},
			ymin = 0.5,
			ymax = 8,
			legend pos=south east,
			xmajorgrids=true,
			ymajorgrids=true,
			grid style=dashed,
			ytick = {1,1.5,...,7.5},
			xlabel = $Threads$, 
			ylabel = $Speedup$]
			\addplot table [domain=1:16, samples=1000,x=threads, y=GRID_speedup, col sep=comma] {times_openmp_skalierung_1.csv};
			\addlegendentry{$linked-cells-1$}
			\addplot table [domain=1:16, samples=1000,x=threads, y=GRID_speedup, col sep=comma] {times_openmp_skalierung_1_2.csv};
			\addlegendentry{$linked-cells-1.2$}
			\addplot table [domain=1:16, samples=1000,x=threads, y=GRID_LIST_speedup, col sep=comma] {times_openmp_skalierung_1.csv};
			\addlegendentry{$linked-cells+verlet-list-1$}
			\addplot table [domain=1:16, samples=1000,x=threads, y=GRID_LIST_speedup, col sep=comma] {times_openmp_skalierung_1_2.csv};
			\addlegendentry{$linked-cells+verlet-list-1.2$}
			\end{axis}
			\end{tikzpicture}
		\end{center}
	\end{figure}
\end{frame}
\section{Laufzeitverhalten}
\subsection{}
\begin{frame}
	\frametitle{Nachbar-Listen}
	\begin{columns}
		\column{0.6\linewidth}
		\centering
			\includegraphics[width=1\textwidth]{List_Zoom_1.png}
		\column{0.4\linewidth}
		\begin{itemize}
			\item Aufbau\\ $\Theta\left(p^2\right)$
			\item Iteration \\$O\left(p^2\cdot \frac{\frac{4}{3}\pi\cdot r^3}{V}\right)$
		\end{itemize}
	\end{columns} 
p $\rightarrow$ alle Partikel\\
r $\rightarrow$ cut-off-radius \\
V $\rightarrow$ gesamt Volumen
\end{frame}
\begin{frame}
	\frametitle{Linked-Cells}
	\begin{columns}
		\column{0.6\linewidth}
		\centering
		\includegraphics[width=1\textwidth]{Grid_Zoom_1.png}
		\column{0.4\linewidth}
		\begin{itemize}
			\item Aufbau\\ $\Theta\left(p\right)$
			\item Iteration \\$O\left(p^2\cdot \frac{27 \cdot r^3}{V}\right)$
		\end{itemize}
	\end{columns} 
p $\rightarrow$ alle Partikel\\
r $\rightarrow$ cut-off-radius \\
V $\rightarrow$ gesamt Volumen
\end{frame}
\begin{frame}
	\frametitle{Linked-Cells + Nachbar-Listen}
	\begin{columns}
		\column{0.6\linewidth}
		\centering
		\includegraphics[width=1\textwidth]{Grid_List_Zoom_1.png}
		\column{0.4\linewidth}
		\begin{itemize}
			\item 1.Aufbau (Linked-Cells) \\ $\Theta\left(p\right)$
			\item 2.Aufbau (Nachbar-Listen) \\$O\left(p^2\cdot \frac{27 \cdot r^3}{V}\right)$
			\item Iteration \\$O\left(p^2\cdot \frac{\frac{4}{3}\pi\cdot r^3}{V}\right)$
		\end{itemize}
	\end{columns} 
p $\rightarrow$ alle Partikel\\
r $\rightarrow$ cut-off-radius \\
V $\rightarrow$ gesamt Volumen
\end{frame}
\begin{frame}
	\frametitle{Gegenüberstellung(1)}
	\begin{itemize}
		\item Simulation für \textbf{Langreichweitige} Interaktionen \\
		$\rightarrow p\cdot\frac{r^3}{V}\sim p$
		%der zu betrachende raum pro partikel ist relativ groß
	\end{itemize}
	\begin{tabular}{c|l|l|l}
		& Nachbar-Listen & Linked-Cells & Kombination \\
		\hline
		\begin{tabular}{@{}c@{}}Aufbau \\ (Linked-Cells)\end{tabular} & - & $\Theta\left(p\right)$& $\Theta\left(p\right)$\\
		\hline
		\begin{tabular}{@{}c@{}}Aufbau \\ (Nachbar-Listen)\end{tabular}& $\Theta\left(p^2\right)$ & - & $O\left(p^2\cdot \frac{27 \cdot r^3}{V}\right)$ \\
		\hline
		Iteration&$O\left(p^2\cdot \frac{\frac{4}{3}\pi\cdot r^3}{V}\right)$ &$O\left(p^2\cdot \frac{27 \cdot r^3}{V}\right)$ & $O\left(p^2\cdot \frac{\frac{4}{3}\pi\cdot r^3}{V}\right)$ \\
	\end{tabular}
\end{frame}
\begin{frame}
	\frametitle{Gegenüberstellung(2)}
	\begin{itemize}
		\item Simulation für \textbf{Kurzreichweitige} Interaktionen \\
		$\rightarrow p\cdot\frac{r^3}{V}\sim 1$
		%der zu betrachtede raum pro partikel ist relativ klein
	\end{itemize}
	\begin{tabular}{c|l|l|l}
		& Nachbar-Listen & Linked-Cells & Kombination \\
		\hline
		\begin{tabular}{@{}c@{}}Aufbau \\ (Linked-Cells)\end{tabular} & - & $\Theta\left(p\right)$& $\Theta\left(p\right)$\\
		\hline
		\begin{tabular}{@{}c@{}}Aufbau \\ (Nachbar-Listen)\end{tabular}& $\Theta\left(p^2\right)$ & - & $O\left(p\cdot 27\right)$ \\
		\hline
		Iteration&$O\left(p\cdot \frac{4}{3}\pi\right)$ &$O\left(p\cdot 27\right)$ & $O\left(p\cdot \frac{4}{3}\pi\right)$ \\
	\end{tabular}
\end{frame}
\section{Autotuneing}
\subsection{}
\begin{frame}[t]
	\frametitle{Laufzeitmessungen}
	\begin{figure}
		\begin{center}
			\begin{tikzpicture}
			\begin{axis}[
				xmin = 0,
				xmax = 25,
				xtick = {1,...,24},
				ymin = 100,
				ymax = 320,
				legend pos=north west,
				ymajorgrids=true,
				grid style=dashed,
				ytick = {100,120,...,320},
				xlabel = $Parameter-Kombination$, 
				ylabel = $Sekunden$]
				\addplot table [domain=1:24, only marks, samples=1000,x=row, y=GRID_average_time, col sep=comma] {times_gleichverteilung.csv};
				\addlegendentry{$linked-cells$}
				\addplot table [domain=1:24, only marks, samples=1000,x=row, y=GRID_LIST_average_time, col sep=comma] {times_gleichverteilung.csv};
				\addlegendentry{$linked-cells+verlet-list$}
			\end{axis}
			\end{tikzpicture}
		\end{center}
	\end{figure}
\end{frame}
\begin{frame}[t]
	\frametitle{Resultierende Entscheidung}
	\begin{itemize}
		\item \textbf{c} $\rightarrow$ cut-off-radius
		\item \textbf{f} $\rightarrow$ cut-off-radius-factor
		\item \textbf{s} $\rightarrow$ start-speed
	\end{itemize}
	\begin{align*}
		2\cdot s < c \cdot (f - 1) - 1
	\end{align*}
	\begin{itemize}
		\item \textbf{true} $\rightarrow$ linked-cells+verlet-list
		\item \textbf{false} $\rightarrow$ linked-cells
	\end{itemize}
\end{frame}
\section{Literatur}
\subsection{}
\begin{frame}
	\frametitle{Literatur}
	\begin{itemize}
        \item M-Griebel, S. Knapek, G. Zumbuschm, A. Caglar: Numerische Simulation in der Moleküldynamic. Springer, 2003
		\item D.C Rapaport: The Art of Molecular Dynamics Simulation - 2nd edition, Cambridge University Press, 2004
	\end{itemize}
\end{frame}

\end{document}
