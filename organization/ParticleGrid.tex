\documentclass{article}
\usepackage[a4paper,left=35mm,top=26mm,right=26mm,bottom=15mm]{geometry}
\usepackage[utf8]{inputenc}
\usepackage{lineno}
\usepackage{listings}
\usepackage{color}
\usepackage{caption}
\usepackage{listings}
\usepackage{amsmath}
\usepackage{pxfonts}
\usepackage{pgfplots}
 \usepackage{pgfplotstable}
 \usepackage{hyperref}
 \usepackage{filecontents}
\title{ParticleGrid}
\author{Benjamin Warnke}
\date{\today}

\lstset{language=C,
    basicstyle=\ttfamily,
    keywordstyle=\bfseries,
    showstringspaces=false,
    morekeywords={include, printf}
}

\begin{document}

\maketitle

\section*{Skizze 1}

\setlength{\unitlength}{50pt}

\begin{picture}(0,0)
\linethickness{1pt} 
\put(0,-0){\line(+9,0){9}}
\put(0,-1){\line(+9,0){9}}
\put(0,-2){\line(+9,0){9}}
\put(0,-3){\line(+9,0){9}}
\put(0,-4){\line(+9,0){9}}
\put(0,-5){\line(+9,0){9}}
\put(0,-6){\line(+9,0){9}}
\put(0,-7){\line(+9,0){9}}
\put(0,-8){\line(+9,0){9}}
\put(0,-9){\line(+9,0){9}}
\put(+0,0){\line(0,-9){9}}
\put(+1,0){\line(0,-9){9}}
\put(+2,0){\line(0,-9){9}}
\put(+3,0){\line(0,-9){9}}
\put(+4,0){\line(0,-9){9}}
\put(+5,0){\line(0,-9){9}}
\put(+6,0){\line(0,-9){9}}
\put(+7,0){\line(0,-9){9}}
\put(+8,0){\line(0,-9){9}}
\put(+9,0){\line(0,-9){9}}
\linethickness{4pt} 
\put(4.5,-4.5){\line(0,+1){1}}
\put(4.5,-4.5){\line(1,+1){1}}
\put(4.5,-4.5){\line(1,+0){1}}
\put(4.5,-3.5){\line(1,-1){1}}
\end{picture}
\newpage










\section*{Lennard Jones}
Siehe Rapport The Art of Molecular Dynamics Simulation Seite 12 unten.\\
\begin{itemize}
	\item $i,j$ Partikel Indices
	\item $r_{ij}$ Abstand der Partikel i und j
	\item $\sigma$ ???
	\item $\epsilon$ ???
\end{itemize}
\begin{align*}
	f_{ij}&=\left(\frac{48\epsilon}{\sigma^2}\right)\left[\left(\frac{\sigma}{r_{ij}}\right)^{14}-\frac{1}{2}\left(\frac{\sigma}{r_{ij}}\right)^8\right]r_{ij}
\end{align*}
\section*{Verlet Algorithmus}
\begin{itemize}
	\item $n$ Zeitschritt Nummer
	\item $x_{n}$ Position zum Zeitpunkt n
	\item $\Delta t$ Zeitschritt Größe
\end{itemize}
Siehe Wikipedia \url{https://de.wikipedia.org/wiki/Verlet-Algorithmus}\\
\begin{align*}
	\vec{x}_1&=\vec{x}_0+\vec{v}_0t+\frac{1}{2}\vec{a}_0\Delta t^2\\
	\vec{x}_{n+1}&=2\vec{x}_n-\vec{x}_{n-1}+\vec{a}_n\Delta t^2
\end{align*}

\section*{Kraft $\leftrightarrow$ Beschleunigung}
\begin{align*}
	f&=ma\\
	a&=\frac{f}{m}
\end{align*}

\section*{gerichtete Kraft von i nach j}
\begin{itemize}
	\item $x_{i}$ Position des Partikels i
	\item $x_{j}$ Position des Partikels j
	\item $a$ Kraft
\end{itemize}
Achtung Normieren der Richtung !!\\
\begin{align*}
	\vec{a}=\frac{a}{r_{ij}}\left(\vec{x}_{j}-\vec{x}_{i}\right)
\end{align*}
\newpage
\section*{Alles Zusammen}
\begin{align}
	\vec{x}_{n+1}&=2\vec{x}_n-\vec{x}_{n-1}+\vec{a}_n\Delta t^2\\
	&=2\vec{x}_{i,n}-\vec{x}_{i,n-1}+\frac{a_n}{r_{n,ij}}\left(\vec{x}_{n,j}-\vec{x}_{n,i}\right)\Delta t^2\\
	&=2\vec{x}_{i,n}-\vec{x}_{i,n-1}+\frac{\frac{f_{n,ij}}{m_i}}{r_{n,ij}}\left(\vec{x}_{n,j}-\vec{x}_{n,i}\right)\Delta t^2\\
	&=2\vec{x}_{i,n}-\vec{x}_{i,n-1}+\frac{f_{n,ij}}{m_ir_{n,ij}}\left(\vec{x}_{n,j}-\vec{x}_{n,i}\right)\Delta t^2\\
	&=2\vec{x}_{i,n}-\vec{x}_{i,n-1}+\frac{\left(\frac{48\epsilon_{n,ij}}{\sigma_{n,ij}^2}\right)\left[\left(\frac{\sigma_{n,ij}}{r_{n,ij}}\right)^{14}-\frac{1}{2}\left(\frac{\sigma_{n,ij}}{r_{n,ij}}\right)^8\right]r_{n,ij}}{m_ir_{n,ij}}\left(\vec{x}_{n,j}-\vec{x}_{n,i}\right)\Delta t^2\\
	&=2\vec{x}_{i,n}-\vec{x}_{i,n-1}+\frac{\left(\frac{48\epsilon_{n,ij}}{\sigma_{n,ij}^2}\right)\left[\left(\frac{\sigma_{n,ij}}{r_{n,ij}}\right)^{14}-\frac{1}{2}\left(\frac{\sigma_{n,ij}}{r_{n,ij}}\right)^8\right]}{m_i}\left(\vec{x}_{n,j}-\vec{x}_{n,i}\right)\Delta t^2\\
	&=2\vec{x}_{i,n}-\vec{x}_{i,n-1}+\frac{\left(\frac{48\epsilon_{n,ij}}{\sigma_{n,ij}^2}\right)\left[\left(\frac{\sigma_{n,ij}^{14}}{r_{n,ij}^{14}}\right)-\frac{1}{2}\left(\frac{\sigma_{n,ij}^8}{r_{n,ij}^8}\right)\right]}{m_i}\left(\vec{x}_{n,j}-\vec{x}_{n,i}\right)\Delta t^2\\
	&=2\vec{x}_{i,n}-\vec{x}_{i,n-1}+\frac{\left(48\epsilon_{n,ij}\sigma_{n,ij}^{6}\right)\left[\left(\frac{\sigma_{n,ij}^{6}}{r_{n,ij}^{14}}\right)-\left(\frac{0.5}{r_{n,ij}^8}\right)\right]}{m_i}\left(\vec{x}_{n,j}-\vec{x}_{n,i}\right)\Delta t^2\\
\end{align}


\section*{Datenstrukturen}
\begin{itemize}
	\item Partikel
	\begin{itemize}
		\item letzte     Position
		\item aktuelle Position
		\item nächste Position
		\item Partikel-Typ
	\end{itemize}
	\item Partikel-Typ
	\begin{itemize}
		\item Masse
		\item $\sigma$ in Verbindung mit jedem beliebigen anderen Partikel-Typ
		\item $\epsilon$ in Verbindung mit jedem beliebigen anderen Partikel-Typ
	\end{itemize}
\end{itemize}






\end{document}
