%url = https://www.overleaf.com/6868875yrhfvynywgzr
%url = https://git.overleaf.com/6868875yrhfvynywgzr


\documentclass[a4paper]{article}
%% Language and font encodings
\usepackage[english]{babel}
\usepackage[utf8x]{inputenc}
\usepackage[T1]{fontenc}

%% Sets page size and margins
\usepackage[a4paper,top=3cm,bottom=2cm,left=3cm,right=3cm,marginparwidth=1.75cm]{geometry}

%% Useful packages
\usepackage{amsmath}
\usepackage{graphicx}
\usepackage[colorinlistoftodos]{todonotes}
\usepackage[colorlinks=true, allcolors=blue]{hyperref}

\title{Universität Hamburg Projekt 2016 Particle Simulation}
\author{Olliver Heidmann Benjamin Warnke}

\begin{document}
\maketitle

\section{Projektplan}
\begin{enumerate}
\item startparameter (Oliver)
    \begin{itemize}
        	\item verbose
            \item algorithmus
            \item autotuneing on/off
            \item timestep größe
            \item abstand der datei speicherung
            \item optimierungsmethode festlegen (nur ohne autotuneing)
            \item seed für random
        \end{itemize}
\item prototypen/interface für Optimierte Datenstrukturen(Oliver)
	\begin{itemize}
        \item  insert(Particle):void
        \item  next():Particle
        \item  neighbours(Particle):Particlelist
        \item  zusätzliche? weniger?
        \item  dies würde die austauschbarkeit enorm vereinfachen ... autotuneing müsste nur noch eine variable ändern
        \item  gemeinsam treffen, nachdem die vorherigen aufgaben abgeschlossen sind.
    \end{itemize}
\item Debug+Benchmark funktionen (Benjamin)
\item importieren/generieren der Startdaten (Benjamin)
	\begin{itemize}
        \item aus Datei (Optional)
        \item  generieren nach bestimmten verteilungen für tests, welches verfahren wann am besten ist
        \begin{itemize}
        	\item  kugel in der mitte
        	\item  verteilte kugeln jeweils gleichmäßig gefüllt
        	\item  gleichmäßig
        	\item  pseudorandom
        	\item  wenige partikel
        	\item  viele partikel
        	\item  weit auseinander
        	\item  eng zusammen
        	\item  generieren nach mustern
        \end{itemize}
    \end{itemize}
\item implementierung der Lennard-Jones-Simulation (Oliver)
	\begin{itemize}
        \item lesen im kapitel des buches?!?
        \item was für daten werden benötigt?
        \begin{itemize}
        	\item distanz zum anderen
        \end{itemize}
    \end{itemize}
\item ausgabe der Daten (Benjamin)
	\begin{enumerate}
		\item 
        \begin{itemize}
        	\item eigenes txt format für tests
            \item exportieren in bekannte Dateiformate (mindestens 1)
			\begin{itemize}
	            \item LAMMPS
	            \item ESPRESSO
	            \item GROMACS
	            \item VMD
	            \item ParaView/vtk
	        \end{itemize}
        \end{itemize}
		\item
        \begin{itemize}
        	\item OpenGl (optioal)
        \end{itemize}
    \end{enumerate}
\item autotuneing + analyse
	\begin{itemize}
        \item als Grid (Benjamin)
        \item mit Listen der Nachbarn (Oliver)
        \item gruppen zusammenfassen und nur den Schwerpunkt berechnen
        \item welches verfahren sollte "gelöscht" werden -> wiso?
        \item welches verfahren ist am besten -> wiso? abhängig von der eingabe??
        \item kriterien für analyse?
        \item eingaben kategorisieren um analysieren zu können
    \end{itemize}
\item optimierung
	\begin{itemize}
        \item funtions Verfahren
		\item openmp
        \item Intel TBB (optioal)
        \item cuda (optional) 
        \item opencl (optional)
        \item mpi (optional)
    \end{itemize}
\item andere Simulationsverfahren
	\begin{itemize}
        \item Smoothed Particle Hydrodynamics
        \item Dissipative Particle Dynamics
    \end{itemize}
\end{enumerate}

\end{document}
